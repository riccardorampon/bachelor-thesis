L'apprendimento Multi-label è una delle classi di apprendimento che ammette per ogni istanza l'assegnazione di più più labels contemporaneamente. Il lavoro su cui è basato questo elaborato propone un nuovo insieme di metodi per la classe di problemi rappresentata dalla classificazione Multi-label, nei quali il nucleo dell'approccio proposto prevede di combinare tra loro un insieme di Gated Recurrent Units (GRU) e Temporal Convolutional Neural Networks (TCN) con l'utilizzo di una nuove varianti rispetto al metodo di ottimizzazione Adam (Adaptive Momentum stimation). Inoltre, le reti neurali proposte vengono combinate con l'approccio IMCC (Incorporating Multiple Clustering Centers) che rappresenta l'attuale stato dell'arte in merito alla classificazione Multi-label. Gli esperimenti effettuati dimostrano la potenza di questo approccio, che ha dimostrato di superare i migliori metodi riportati in letteratura.